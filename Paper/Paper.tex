\documentclass[12pt, final]{article}
\usepackage{color}
\usepackage{times}
\usepackage{amssymb,amsmath,amsthm}
\usepackage{comment}
\usepackage{dsfont}
\usepackage{graphicx}
\usepackage{enumerate}
\usepackage{enumitem}
\usepackage[paperwidth=8.5in,left=1.0in,right=1.0in,top=1.0in,bottom=1.0in,paperheight=11.0in]{geometry}
\usepackage[labelsep=colon,singlelinecheck=false,footnotesize]{caption}
\usepackage{fancyhdr}
\usepackage{float}
\usepackage{booktabs}
\usepackage[sort&compress]{natbib}
\usepackage{subfig}
\usepackage{titlesec}
\usepackage[breaklinks]{hyperref}
\hypersetup{pdfdisplaydoctitle=true,bookmarksnumbered=true,colorlinks=true,citecolor=black,linkcolor=darkblue,urlcolor=darkred,pdfstartview=FitH,pdfpagemode=UseNone}
\usepackage[hyphenbreaks]{breakurl}
\usepackage{accents}
\newcommand{\ubar}[1]{\underaccent{\bar}{#1}}


%Define Color Names
\definecolor{Gray}{rgb}{0.65,0.65,0.65}
\definecolor{darkblue}{rgb}{0,0,0.55}
\definecolor{darkred}{rgb}{0.5,0,0}

%Section Headings
\newcommand\Bheadfont{\fontsize{14pt}{\baselineskip}\selectfont}
\titleformat{\section}[hang]{\normalfont\sc\color{darkblue}\Bheadfont}{\thesection\hskip0.618em}{0em}{}
\titlespacing*{\section}{0pt}{15pt plus 2pt minus 2pt}{9pt plus 2pt minus 2pt}
\titleformat{\subsection}[runin]{\normalfont\sc\color{darkblue}} {\thesubsection\hskip0.618em}{0em}{}
\titlespacing*{\subsection}{0pt}{13pt plus 2pt minus 2pt}{13pt plus 2pt minus 2pt}
\titleformat{\subsubsection}[runin]{\normalfont\sc\color{darkblue}} {\thesubsubsection\hskip0.618em}{0em}{}
\titlespacing*{\subsubsection}{0pt}{13pt plus 2pt minus 2pt}{13pt plus 2pt minus 2pt}
\titleformat{\paragraph}[runin]{\bfseries}{\theparagraph\hskip0.618em}{0em}{}
\titlespacing*{\paragraph}{0pt}{13pt plus 2pt minus 2pt}{13pt plus 2pt minus 2pt}

\begin{document}

\section{Model}
\subsection{Firms} The production sector consists of a continuum of monopolistically competitive intermediate goods firms and a final goods firm. Intermediate firm $f \in [0,1]$ produces a differentiated good, $y(f)$, according to $y_t(f) = (k_{t-1}(f))^\alpha(z_tn_t(f))^{1-\alpha}$, where $n(f)$ is the labor hired by firm $f$ and $k(f)$ is the capital rented by firm $f$. $z_t = g_tz_{t-1}$ is technology, which is common across firms. Deviations from the steady-state growth rate, $\bar{g}$, follow
\begin{gather}
  \label{eq:1}
  g_t = \bar{g} + \sigma_g\varepsilon_{g,t},\; \varepsilon_g \sim \mathds{N}(0,1). 
\end{gather}

The final goods firm purchases output from each intermediate firm to produce the final good, $y_t \equiv [\int_{0}^1 y_t(f)^{(\theta-1)/\theta}df]^{\theta/(\theta-1)}$, where $\theta > 1$ is the elasticity of substitution. Dividend maximization determines the demand for intermediate good $f$, $y_t(f) = (p_t(f)/p_t)^{-\theta}y_t$, where $p_t = [\int_{0}^{1} p_t(f)^{1-\theta}df]^{1/(1-\theta)}$ is the price level. Following Rotemberg (1982), intermediate firms pay a price adjustment cost, $adj_t^p(f) \equiv \varphi(p_t(f)/(\bar{\pi}p_{t-1}(f))-1)^2)y_t/2$, where $\varphi > 0$ scales the cost and $\bar{\pi}$ is the steady-state gross inflation rate. Given this cost, firm $f$ chooses $n_t(f)$, $k_{t-1}(f)$, and $p_t(f)$ to maximize the expected discounted present value of future dividends, $E_t\sum_{k=t}^\infty q_{t,k}d_k(f)$, subject to its production function and the demand for its product, where $q_{t,t} \equiv 1$, $q_{t,t+1} \equiv \beta(\lambda_t/\lambda_{t+1})$ is the pricing kernel between periods $t$ and $t+1$, $q_{t,k} \equiv \prod_{j=t+1}^{k>t} q_{j-1,j}$, and $d_t(f) = p_t(f)y_t(f)/p_t - w_tn_t(f) - adj_t^p(f)$. In symmetric equilibrium, the optimality conditions reduce to
\begin{gather}
  y_t = (k_{t-1})^\alpha(z_tn_t)^{1-\alpha},\\
  w_t = (1-\alpha)mc_ty_t/n_t,\\
  r_t^k = \alpha mc_t y_t/k_{t-1},\\
  \varphi(\pi_t^{gap}-1)\pi_t^{gap} = 1-\theta + \theta mc_t + \beta\varphi E_t[(\lambda_t/\lambda_{t+1})(\pi_{t+1}^{gap}-1)\pi_{t+1}^{gap}(y_{t+1}/y_t)],
\end{gather}
where $\pi^{gap}_t = \pi_t/\bar{\pi}_t$ and $\pi_t = p_t/p_{t-1}$ is the gross inflation rate. If $\varphi = 0$, the real marginal cost of producing a unit of output ($mc_t$) equals $(\theta-1)/\theta$, which is the inverse of the markup of price over marginal cost.
\subsection{Households} The households choose $\{c_t, n_t, b_t, x_t, k_t\}_{t=0}^\infty$ to maximize expected lifetime utility given by $E_0\sum_{t=0}^\infty\beta[\log(c_t-hc^a_{t-1}) - \chi n_t^{1+\eta}/(1+\eta)]$, where $\beta$ is the discount factor, $\chi > 0$ determines steady-state labor, $1/\eta$ is the Frisch elasticity of labor supply, $c$ is consumption, $c^a$ is aggregate consumption, $h$ is the degree of external habit persistence, $b$ is the real value of a privately-issued 1-period nominal bond, $x$ is investment, and $E_0$ is an expectation operator conditional on information available in period 0. The household's budget constraint is given by
\begin{gather*}
    c_t+x_t+b_t/(i_ts_t)=w_tn_t+r_t^kk_{t-1}+b_{t-1}/\pi_t+d_t
  \end{gather*}
where $i$ is the gross nominal interest rate, $r^k$ is the capital rental rate, and $d$ is a real dividend from ownership of intermediate firms. The nominal bond, $b$ is subject to a risk premium, $s$, that follows
\begin{gather}
  \label{eq:6}
  s_t = (1-\rho_s)\bar{s} + \rho_ss_{t-1} + \sigma_s\varepsilon_{s,t},\; 0 \leq \rho_s < 1,\; \varepsilon_s \sim \mathds{N}(0,1),
\end{gather}
where $\bar{s}$ is the steady-state value. An increase in $s_t$ boosts saving, which lowers period-$t$ demand.

Households also face an investment adjustment cost, so the law of motion for capital is given by
\begin{gather}
  k_t = (1-\delta)k_{t-1} + x_t(1-\nu(x^g_t - 1)^2/2),\; 0 \leq \delta \leq 1,
  \end{gather}
where $x_t^g = x_t/(\bar{g}x_{t-1})$ is investment growth relative to its steady-state and $\nu \geq 0$ scales the cost.

The first order conditions to each household's constrained optimization problem are given by
\begin{gather}
  \lambda_t = c_t - hc^a_{t-1}, \\
  w_t = \chi n_t^\eta \lambda_t,\\
  1 =  \beta E_t[(\lambda_t/\lambda_{t+1})(s_ti_t/(\bar{\pi}\pi_{t+1}^{gap}))],\\
  q_t = \beta E_t[(\lambda_t/\lambda_{t+1})(r^k_{t+1}+(1-\delta)q_{t+1})],\\
  1 = q_t[1-\nu(x^g_t-1)^2/2 - \nu(x_t^g-1)x_t^g] + \nu\beta\bar{g}E_t[q_{t+1}(\lambda_t/\lambda_{t+1})(x^g_{t+1})^2(x^g_{t+1}-1)],\\
  \varphi(\pi_t^{gap}-1)\pi_t^{gap} = 1-\theta + \theta mc_t + \beta\varphi E_t[(\lambda_t/\lambda_{t+1})(\pi_{t+1}^{gap}-1)\pi_{t+1}^{gap}(y_{t+1}/y_t)],
\end{gather}
where $1/\lambda$ is the marginal utility of consumption and $q$ is Tobin's q.\\

\noindent
\textbf{Monetary Policy} The central bank sets the gross nominal interest rate, $i$, according to
\begin{gather}
    \label{eq:15}
    i_t=\max\{1,i_t^n\}\\
      \label{eq:16}
  i_t^n=(i^n_{t-1})^{\rho_i}(\bar{\imath}(\pi^{gap}_t)^{\phi_\pi}(y^{g}_{t})^{\phi_y})^{1-\rho_i}\exp(\sigma_i\varepsilon_{i,t}), 0 \leq \rho_i < 1, \varepsilon_i \sim \mathds{N}(0,1),
  \end{gather}
where $y^{gdp}$ is real GDP (i.e., output, $y$, minus the resources lost due to adjustment costs, $adj^p$), $i^n$ is the gross notional interest rate, $\bar i$ and $\bar{\pi}$ are the target values of the inflation and nominal interest rates, and $\phi_\pi$ and $\phi_y$ are the responses to the inflation and output growth gaps. A more negative net notional rate indicates that the central bank is more constrained.\\

\noindent
\textbf{Competitive Equilibrium} The aggregate resource constraint and real GDP definition are given by
\begin{gather}
  c_t + x_t = y_t^{gdp}\\
  y_t^{gdp} = [1 - \varphi(\pi_t^{gap}-1)^2/2]y_t
\end{gather}
The model does not have a steady-state due to the unit root in technology, $z_t$. Therefore, we define the variables with a trend in terms of technology (i.e., $\tilde{x}_t \equiv x_t/z_t$). The detrended equilibrium system is provided in \hyperlink{Appendix A}{Appendix A}. A competitive equilibrium consists of sequences of quantities, $\{\tilde{c}_t, \tilde{y}_t, \tilde{y}_t^{gdp}, x^g_t, y^g_t, n_t, \tilde{k}_t, \tilde{x}_t\}_{t=0}^\infty$, prices, $\{\tilde{w}_t, i_t, i^n_t, \pi_t, \tilde{\lambda}_t, q_t, r^k_t, mc_t\}_{t=0}^\infty$, and exogenous variables, $\{s_t,g_t\}_{t=0}^\infty$, that satisfy the detrended equilibrium system, given the initial conditions, $\{\tilde{c}_{-1}, i^n_{-1}, \tilde{k}_{-1}, \tilde{x}_{-1}, \tilde{w}_{-1}, s_0, g_0, \varepsilon_{i,0}\}$, and three sequences of shocks, $\{\varepsilon_{g,t}, \varepsilon_{s,t}, \varepsilon_{i,t}\}_{t=1}^\infty$.

\subsection{Parameter Values} \hyperlink{Table 1}{Table 1} shows the true model parameters. The parameters were chosen so our data generating process is characteristic of U.S. data. The steady-state growth rate ($\bar{g}$), inflation rate ($\bar{\pi}$), risk-premium ($\bar{s}$), and capital share of income ($\alpha$) are equal to the time averages of per capital real GDP growth, the percent change in the GDP implicit price deflator, the Baa corporate bond yield relative to the yield on the 10-Year Treasury, and the Fernald (2012) utilization-adjusted quarterly-TFP estimates of the capital share of income from 1988Q1-2017Q4.

The subjective discount factor, $\beta$, is set to $0.9949$, which is the time average of the values implied by the steady-state consumption Euler equation and the federal funds rate. The corresponding annualized steady-state nominal interest rate is $3.3\%$ which is consistent with the sample average and current long-run estimates of the federal funds rate. The leisure preference parameter, $\chi$, is set to steady-state labor equals $1/3$ of the available time. The elasticities of substitution between intermediate goods $\theta$, is set to $6$, which corresponds to a $20\%$ average markup and match the values used in Gust et al. (2017). The Frisch elasticity of labor supply, $1/\eta$, is set to $3$ to match the macro estimate in Peterman (2016). The remainin parameters are set to round numbers that are in line with the posterior estimates from similar models in the literature.

\begin{table}
  \hypertarget{Table 1}
  \small
    \setlength{\tabcolsep}{10pt}      
  \begin{tabular}{l c c l c c}
    \hline
  Subjective Discount Factor & $\beta$ & $0.9949$ & Rotemberg Price Adjustment Cost & $\varphi$ & $100$ \\
   Frisch Labor Supply Elasticity & $1/\eta$ & $3$ & Inflation Gap Response & $\phi_\pi$ & $2.0$ \\
  Price Elasticity of Substitution & $\theta$ & $6$ & Output Growth Gap Response & $\phi_y$ & $0.5$ \\
  Steady-State Labor Hours & $\bar{n}$ & $1/3$ & Habit Persistence & $h$ & $0.80$ \\
  Steady-State Risk Premium & $\bar{s}$ & $1.0058$ & Risk Premium Persistence & $\rho_s$ & $0.80$ \\
  Steady-State Growth Rate & $\bar{g}$ & $1.0034$ & Notional Rate Persistence & $\rho_i$ & $0.80$ \\
  Steady-State Inflation Rate & $\bar{\pi}$ & $1.0053$ & Technology Growth Shock SD & $\sigma_g$ & $0.005$ \\
  Capital Share of Income & $\alpha$ & $0.35$ & Risk Premium Shock SD & $\sigma_s$ & $0.005$ \\
  Capital Depreciation Rate & $\delta$ & $0.025$ & Notional Interest Rate Shock SD & $\sigma_i$ & $0.0035$ \\
  Investment Adjustment Cost & $\nu$ & $4$ & & &\\
  \hline
  \end{tabular}
  \caption{Parameter values}
  \label{Tab:1}
\end{table}

\section{Solution Methods}
\subsection{Richter et al. (2014)}
We consider the solution methods of Richter et al. (2014) and Gust et al. (2017). The Richter et al. (2014) solution method is policy function iteration with time iteration and linear interpolation, which is based on the theoretical work on monotone operators in Coleman (1991). We discretize the endogenous state variables and approximate the exogenous states $s_t$, $g_t$, and $\varepsilon_{i,t}$ using the N-state Markov chain in Rouwenhorst (1995). We use the Rowenhorst method so that we only have to interpolate along the dimensions of the endogenous state variables, allowing the solution to be faster and more accurate than quadrature methods. To obtain initial conjectures for the nonlinear policy functions, we solve the level-linear version of our model with Sims's (2002) gensys algorithm. Next, we update the policy functions on each node using a fixed point iteration scheme and compute the maximum distance between the updated policy functions and the inital conjectures. Finally, we replace the initial conjectures with the updated policy functions and iterate until the maximum distance is below the tolerance level. \hyperlink{Appendix B}{Appendix B} provides a more detailed discussion of the solution method. %PLEASE REWRITE THIS SO YOU ARE NOT COPYING
%Add Gust et al solution method
\subsection{Gust et al. (2017)}
The Gust et al. (2017) solution method similarly uses time iteration on a fixed point solution with linear interpolations. Following this solution method, instead of directly computing the policy functions, we estimate smoother functions at and away from the ZLB following Gust, L{\o g}pez-Salido, and Smith (2012) which builds on Christiano and Fisher (2000). Since the policy functions depend directly on the nominal interetst rate, they have a kink or non-differentiability associated with the ZLB.  On the other hand, the regime-indexed policy functions do not depend on the current indicator function and are thus more likely to be smooth. The smoother functions are approximated by specifying:
\begin{gather}
  \textbf{pf}_t(d) = \textbf{pf}_{t,1}(d)\mathds{I}_t(d) + \textbf{pf}_{t,2}(d)(1-\mathds{I}_t(d))
\end{gather}
where $\mathds{I}_t(d)$ is defined by
\begin{gather}
  \mathds{I}_t(d) = 
\begin{cases}
     1 &\text{ if } i_t > 1\\
     0 &\text{ otherwise}
\end{cases}
\end{gather}
The variable $i_t=\max\{1,i_t^n\}$ represents the value of the notional rate derived from evaluating the functions $\textbf{pf}_{t,j}(d)$ for $j \in \{1,2\}$ and using equation (\ref{eq:16}). (For each variable, $j=1$ denotes the function associated with the regime with a positive nominal rate and $j=2$ denotes the function associated with the ZLB regime.) %In calculating the fixed point, we use $i=1$ when the model is in the ZLB regime. 
The functions $\textbf{pf}_{t,j}$ satisfy the residual functions $R_{t,l,j}$ for and $l \in \{1,2,3,4\}$ which correspond to the household FOC bond, FOC capital, FOC investment, and the price Phillips curve, respectively, and $j = 1,2$.
\begin{gather}
  R_{t,1,1} = 1 - s_ti_t\beta E_t[(\lambda_t/\lambda_{t+1})(1/(\bar{\pi}\pi_{t+1}^{gap}))]\\
  R_{t,1,2} = 1 - s_t\beta E_t[(\lambda_t/\lambda_{t+1})(1/(\bar{\pi}\pi_{t+1}^{gap}))]\\
  R_{t,2,j} = q_t - \beta E_t[(\lambda_t/\lambda_{t+1})(r^k_{t+1}+(1-\delta)q_{t+1})]\\
  R_{t,3,j} = 1 - q_t[1-\nu(x^g_t-1)^2/2 - \nu(x_t^g-1)x_t^g] - \nu\beta\bar{g}E_t[q_{t+1}(\lambda_t/\lambda_{t+1})(x^g_{t+1})^2(x^g_{t+1}-1)]\\
  R_{t,4,j} = \varphi(\pi_t^{gap}-1)\pi_t^{gap} - (1-\theta) - \theta mc_t - \beta\varphi E_t[(\lambda_t/\lambda_{t+1})(\pi_{t+1}^{gap}-1)\pi_{t+1}^{gap}(y_{t+1}/y_t)]
  \end{gather}

  \subsection{Euler Equation Errors}
Richter et al. (2014) use Euler equation errors to measure the accuracy of their solutions. To measure errors between nodes, we use Gauss-Hermite quadrature instead of the Rouwenhorst method to allow exogenous variables to be off the grid. The Euler equation errors are reported in absolute value of the errors in base 10 logarithms. A consumption Euler equation error of -3 means the household makes an error equal to one out of every 1,000 consumption goods. 


\section{Results}

\subsection{Solution Times}
\hyperlink{Table 2}{Table 2} reports the solution times for the Richter et al. (2013) and Gust et al. (2017) solution methods for the models with and without capital. %Say something about the solution times% %Say something about the machine the solution times were computed on

\begin{table}[H]
  \centering
  \hypertarget{Table 2}
  \small
    \setlength{\tabcolsep}{10pt}      
  \begin{tabular}{l c c c c}
    \hline
    & \multicolumn{2}{c}{Model without capital} & \multicolumn{2}{c}{Model with capital} \\
    & Iterations & Total Time & Iterations & Total time\\
    \hline
  Richter et al. (2013) & 67 & 17.9s & 65 & 0h 15m 37.1s\\    
  Gust et al. (2017) & 66 & 46s & 1170 &  5h 32m 48.6s\\
  \hline
\end{tabular}
  \caption{Solution times}
  \label{Tab:2}
\end{table}

\subsection{Policy Functions}


\subsection{Euler Equation Errors}


\appendix
  \section{Detrended Equilibrium System} \hypertarget{Appendix A}
  \noindent
  \textbf{Medium-Scale Model} The detrended system includes (\ref{eq:1}),(\ref{eq:6}),(\ref{eq:15}),(\ref{eq:16}) and 
\begin{gather}
\tilde{y}_t= (\tilde{k}_{t-1}/g_t)^\alpha n_t^{1-\alpha},\\
r^k_t = \alpha mc_t g_t \tilde{y}_t/\tilde{k}_{t-1},\\
\tilde{w}_t = (1-\alpha)mc_t\tilde{y}_t/n_t,\\
\tilde{y}^{gdp}_t = [1-\varphi(\pi_t^{gap} - 1)^2/2]\tilde{y}_t,\\
y^g_t = g_t\tilde{y}^{gdp}_t/(\bar{g}\tilde{y}^{gdp}_{t-1}),\\
\tilde{\lambda}_t = \tilde{c}_t\ - h\tilde{c}_{t-1}/g_t,\\
\label{eq:m1}
\tilde{w}_t = \chi n_t^\eta \tilde{\lambda}_t,\\
\tilde{c}_t + \tilde{x}_t = \tilde{y}^{gdp}_t,\\
x^g_t = g_t\tilde{x}_t/(\bar{g}\tilde{x}_{t-1}),\\
\tilde{k}_t = (1-\delta)(\tilde{k}_{t-1}/g_t) + \tilde{x}_t(1-\nu(x^g_t-1)^2/2),\\%%%
\label{eq:m2}
  1 = \beta E_t[(\tilde{\lambda}_t/\tilde{\lambda}_{t+1})(s_ti_t/(\bar{\pi}\pi_{t+1}^{gap}g_{t+1}))],\\
q_t = \beta E_t[(\tilde{\lambda}_t/\tilde{\lambda}_{t+1})(r^k_{t+1} + (1-\delta)q_{t+1})/g_{t+1}],\\
1 = q_t[1 - \nu(x^g_t-1)^2/2 - \nu(x^g_t-1)x^g_t] + \nu\beta\bar{g}E_t[q_{t+1}(\tilde{\lambda}_t/\tilde{\lambda}_{t+1})(x^g_{t+1})^2(x^g_{t+1}-1)/g_{t+1}],\\
\label{eq:m3}
  \varphi(\pi_t^{gap}-1){\pi}_t^{gap} = 1-\theta + \theta mc_t + \beta\varphi E_t[(\tilde{\lambda}_t/\tilde{\lambda}_{t+1}) (\pi_{t+1}^{gap}-1)\pi_{t+1}^{gap}(\tilde{y}_{t+1}/\tilde{y}_t)].
\end{gather}
The variables are $\tilde{c},\tilde{n},\tilde{x},\tilde{k},\tilde{y^{gdp}},\tilde{y},x^g,y^g,\tilde{w},r^k,\pi,i,i^n,q,mc,\tilde{\lambda},g,$ and $s$.\\

\noindent \textbf{Small-Scale Model} The detrended system includes (\ref{eq:1}),(\ref{eq:6}),(\ref{eq:15}),(\ref{eq:16})(\ref{eq:m1}),(\ref{eq:m2}),(\ref{eq:m3}), and 
\begin{gather}
\tilde{\lambda}_t = \tilde{c}_t,\\ %%%
\tilde{c}_t = [1-\varphi(\pi_t^{gap} - 1)^2/2]\tilde{y}_t,\\ %%%
  \tilde{y}_t= n_t.  %%% 
\end{gather}
The variables are $\tilde{c},i^n,i,\tilde{\lambda},\tilde{w},\pi^{gap},\tilde{y},n,g,$ and $s$.\\ 

\section{Nonlinear Solution Method}\hypertarget{Appendix B}
We begin by compactly writing the detrended nonlinear system as
\begin{gather*}
  E[f(\textbf{s}_{t+1},\textbf{s}_t,\varepsilon_{t+1})|\textbf{z}_t,\vartheta] = 0,
\end{gather*}
where $f$ is a vector-valued function, $\textbf{s}_t$ is a vector of variables, $\varepsilon_t \equiv [\varepsilon_{s,t}, \varepsilon_{g,t},\varepsilon_{i,t}]'$ is a vector of shocks, $\textbf{z}_t$ is a vector of states ($\textbf{z}_t \equiv [\tilde{c}_{-1}, i^n_{t-1},\tilde{k}_{t-1},\tilde{x}_{t-1},s_t,g_t,\varepsilon_{i,t}]'$ for the model with capital and $\textbf{z}_t \equiv [\tilde{c}_{t-1},i^n_{t-1},s_t,g_t,\varepsilon_{i,t}]'$ for the model without capital), and $\vartheta$ is a vector of parameters.

We use the Markov chain method in Rouwenhorst (1995) to discretize the endogenous state variables, $s_t$, $g_t$, and $\varepsilon_{i,t}$. Kopecky and Suen (2010) show the Rouwenhorst method outperforms other methods for approximating autoregressive processes. The bounds on $\tilde{c}_{t-1}, i^n_{t-1}, \tilde{k}_{t-1},$ and $\tilde{x}_{t-1}$ are set to $\pm 2.5\%, \pm 6\%, \pm 8\%, \pm 15\%$, respectively. Richter and Throckmorton chose these bounds so the grids contain 99.9\% of the simulated values for each state variable and during ZLB duration. We discretize the states into 5 evenly-spaced points for the model with capital and 7 evenly-spaced points for the model without capital. The product of the points in each dimension, D, represents the total nodes in the state space ($D = 78125$ for the model with capital and $D = 2401$ for the model without capital). The realization of $\textbf{z}_t$ on node $d$ is denoted $\textbf{z}_t(d)$. The Rouwenhorst method provides integration nodes, $[s_{t+1}(m), g_{t+1}(m), \varepsilon_{i,t+1}(m)]$, with weights, $\phi(m)$, for $m \in \{1, \dots, M\}$. Since the exogenous variables evolve according to a Markov chain, the number of future realizations is the same as the state variables.

The vector of policy functions is denoted $\textbf{pf}_t$ and the realization on node $d$ is denoted $\textbf{pf}_t(d)$ ($\textbf{pf}_t(d) \equiv [\tilde{\pi}^{gap}_t(\textbf{z}_t), n_t(\textbf{z}_t), q_t(\textbf{z}_t), mc_t(\textbf{z}_t)]$ for the model with capital and $\textbf{pf}_t(d) \equiv [\tilde{\pi}^{gap}_t(\textbf{z}_t), \tilde{c}_t(\textbf{z}_t)]$. Our choice of policy functions is not unique, but it helps solving for the other variables in the nonlinear system of equations given $\textbf{z}_t$.

The following steps outline our global policy function iteration algorithm:

\begin{enumerate}
  \item Use Sims's (2002) \texttt{gensys} algorithm to solve the level-linear model without the ZLB constraint. Then map the solution to the discretized state space to initialize the policy functions. 
  \item On each node $d \in \{1,\dots,D\}$, use fixed point iteration to find $\textbf{pf}_t(d)$ to satisfy $E[f(\cdot)|\textbf{z}_t(d), \vartheta] \approx 0$. Guess $\textbf{pf}_t(d) = \textbf{pf}_{j-1}(d)$ Then apply the following: 
    \begin{enumerate}
    \item Solve for all variables dated at time $t$, given $\textbf{pf}_t(d)$ and $\textbf{z}_t(d)$.
    \item Linearly interpolate the policy functions $\textbf{pf}_{j-1}$, at the updated state variables $\textbf{z}_{t+1}(m)$, to obtain $\textbf{pf}_{t+1}(m)$ on every integration node, $m \in \{1,\dots,M\}$.
    \item Given $\{\textbf{pf}_{t+1}(m)\}_{m=1}^M$, solve for the other elements of $\textbf{s}_{t+1}(m)$ and compute%and back out the updated \textbf{pf}$_t(d)$.
      \begin{gather*}
        E[f(\textbf{s}_{t+1},\textbf{s}_t(d),\varepsilon_{t+1})|\textbf{z}_t(d),\vartheta] \approx \sum_{m=1}^M \phi(m)f(\textbf{s}_{t+1}(m),\textbf{s}_t(d),\varepsilon_{t+1}(m)).
      \end{gather*}
      \item Back out $\textbf{pf}_t(d)$ from the expectation operators and updated state.
    \end{enumerate}
       \item Repeat step 2 until $\text{maxdist}_j < 10^{-6}$, where $\text{maxdist}_j \equiv \text{max}\{|\textbf{pf}_j - \textbf{pf}_{j-1}|\}$. When that criterion is satisfied, the algorithm has converged to an approximate nonlinear solution.
\end{enumerate}










\begin{comment}
\subsection{ART Model with capital}
\noindent Preferences:
\begin{gather*}
  E_0\textstyle\sum_{t=0}^\infty\beta^t [\log(c_t-hc^a_{t-1})-\chi n_t^{1+\eta}/(1+\eta)]
\end{gather*}

\noindent Budget Constraint:
\begin{gather*}
  c_t+x_t+b_t/(i_ts_t)=w_tn_t+r_t^kk_{t-1}+b_{t-1}/\pi_t+d_t\\
  k_t = (1-\delta)k_{t-1} + x_t(1 - \nu(x^g_t-1)^2/2)
\end{gather*}

\setcounter{equation}{0}
\noindent Equilibrium system:
\small\begin{gather}
y_t = k_{t-1}^\alpha(z_t n_t)^{1-\alpha}\\ %%%
r_t^k = \alpha mc_ty_t/k_{t-1}\\
w_t = (1-\alpha)mc_ty_t/n_t\\
%w_t^g = \pi_tw_t(\bar{\pi}gw_{t-1})\\
y_t^{gdp} = [1 - \varphi(\pi_t^{gap}-1)^2/2]y_t \\
y^g_t = y^{gdp}_t/(\bar{g}y^{gdp}_{t-1})\\
i_t^n=(i^n_{t-1})^{\rho_i}(\bar{\imath}(\pi^{gap}_t)^{\phi_\pi}(y^{g}_{t})^{\phi_y})^{1-\rho_i}\exp(mp_t)\\
i_t=\max\{1,i_t^n\}\\
\lambda_t = c_t - hc^a_{t-1} \\
w_t = \chi n_t^\eta \lambda_t\\
c_t + x_t = y_t^{gdp}\\
x_t^g = x_t/(\bar{g}x_{t-1})\\
k_t = (1-\delta)k_{t-1}+x_t(1-\nu(x_t^g-1)^2/2) \\%%%
1 =  \beta E_t[(\lambda_t/\lambda_{t+1})(s_ti_t/(\bar{\pi}\pi_{t+1}^{gap}))]\\
q_t = \beta E_t[(\lambda_t/\lambda_{t+1})(r^k_{t+1}+(1-\delta)q_{t+1})]\\
1 = q_t[1-\nu(x^g_t-1)^2/2 - \nu(x_t^g-1)x_t^g] + \nu\beta\bar{g}E_t[q_{t+1}(\lambda_t/\lambda_{t+1})(x^g_{t+1})^2(x^g_{t+1}-1)]\\
\varphi(\pi_t^{gap}-1)\pi_t^{gap} = 1-\theta + \theta mc_t + \beta\varphi E_t[(\lambda_t/\lambda_{t+1})(\pi_{t+1}^{gap}-1)\pi_{t+1}^{gap}(y_{t+1}/y_t)]\\
g_t = \bar{g} + \sigma_g\varepsilon_{g,t}\\
s_t=(1-\rho_s)\bar{s}+\rho_ss_{t-1} + \sigma_s\varepsilon_{s,t}\\
mp_t = \sigma_i\varepsilon_{i,t}\\
  z_t=g_tz_{t-1}
\end{gather}\normalsize
Variables: $\{c,n,x,k,y^{gdp},y,x^g,y^g,w,r^k,\pi,i,i^n,q,mc,\lambda,g,s,mp,z\}$\\

\setcounter{equation}{0}
\noindent De-trended Equilibrium System:
\small\begin{gather}
\tilde{y}_t= (\tilde{k}_{t-1}/g_t)^\alpha n_t^{1-\alpha}\\
r^k_t = \alpha mc_t g_t \tilde{y}_t/\tilde{k}_{t-1}\\
\tilde{w}_t = (1-\alpha)mc_t\tilde{y}_t/n_t\\
\tilde{y}^{gdp}_t = [1-\varphi(\pi_t^{gap} - 1)^2/2]\tilde{y}_t\\
y^g_t = g_t\tilde{y}^{gdp}_t/(\bar{g}\tilde{y}^{gdp}_{t-1})\\
i_t^n=(i^n_{t-1})^{\rho_i}(\bar{\imath}(\pi_t^{gap})^{\phi_\pi}(y^g_t)^{\phi_y})^{1-\rho_i}\exp(\sigma_i\varepsilon_{i,t})\\
i_t=\max\{1,i_t^n\}\\
\tilde{\lambda}_t = \tilde{c}_t\ - h\tilde{c}_{t-1}/g_t\\
\tilde{w}_t = \chi n_t^\eta \tilde{\lambda}_t  \\
\tilde{c}_t + \tilde{x}_t = \tilde{y}^{gdp}_t\\
x^g_t = g_t\tilde{x}_t/(\bar{g}\tilde{x}_{t-1})\\
\tilde{k}_t = (1-\delta)(\tilde{k}_{t-1}/g_t) + \tilde{x}_t(1-\nu(x^g_t-1)^2/2)\\%%%
  1 = \beta E_t[(\tilde{\lambda}_t/\tilde{\lambda}_{t+1})(s_ti_t/(\bar{\pi}\pi_{t+1}^{gap}g_{t+1}))]\\
q_t = \beta E_t[(\tilde{\lambda}_t/\tilde{\lambda}_{t+1})(r^k_{t+1} + (1-\delta)q_{t+1})/g_{t+1}]\\
1 = q_t[1 - \nu(x^g_t-1)^2/2 - \nu(x^g_t-1)x^g_t] + \nu\beta\bar{g}E_t[q_{t+1}(\tilde{\lambda}_t/\tilde{\lambda}_{t+1})(x^g_{t+1})^2(x^g_{t+1}-1)/g_{t+1}]\\
  \varphi(\pi_t^{gap}-1){\pi}_t^{gap} = 1-\theta + \theta mc_t + \beta\varphi E_t[(\tilde{\lambda}_t/\tilde{\lambda}_{t+1}) (\pi_{t+1}^{gap}-1)\pi_{t+1}^{gap}(\tilde{y}_{t+1}/\tilde{y}_t)]\\
  g_t= \bar{g} + \sigma_g\varepsilon_{g,t} \\
  s_t=(1-\rho_s)s_t+\rho_ss_{t-1} + \sigma_s\varepsilon_{s,t}\\
  mp_t = \sigma_i\varepsilon_{i,t}
\end{gather}
Variables:$\{\tilde{c},\tilde{n},\tilde{x},\tilde{k},\tilde{y^{gdp}},\tilde{y},x^g,y^g,\tilde{w},r^k,\pi,i,i^n,q,mc,\tilde{\lambda},g,s,mp\}$\\

\setcounter{equation}{0}
\noindent Log-linear Equilibrium System:
\begin{gather}
  \hat{y}_t/\bar{y} = \alpha(\hat{k}_{t-1}/\bar{k} - \hat{g}_t/\bar{g}) + (1-\alpha)\hat{n}_t/\bar{n}\\
\hat{r}^k_t/\bar{r}^k = \hat{mc}_t/\bar{mc} + \hat{g}_t/\bar{g} + \hat{y}_t/\bar{y} - \hat{k}_{t-1}/\bar{k}\\
\hat{w}_t/\bar{w} = \hat{mc}_t/\bar{mc} + \hat{y}_t/\bar{y} - \hat{n}_t/\bar{n}\\
\hat{y}_t^{gdp} = \hat{y}_t\\
\hat{y}^g_t = \hat{g}_t/\bar{g} + \hat{y}^{gdp}_t/\bar{y}^{gdp} - \hat{y}^{gdp}_{t-1}/\bar{y}^{gdp}\\
  \hat{i}_t/\bar{i} = \rho_i\hat{i}_{t-1}/\bar{i} + (1-\rho_i)(\phi_\pi\hat{\pi}^{gap}_t+ \phi_y\hat{y}^g_t)+\hat{mp}_t \\
  \hat{\imath}_t = \hat{\imath}_t^n\\
  \hat{\lambda}_t = \hat{c}_t - (h/\bar{g})\hat{c}_{t-1} + (h\bar{c}/\bar{g}^2)\hat{g}_t\\
  \hat{c}_t + \hat{x}_t = \hat{y}^{gdp}_t\\
  \hat{x}^g_t = \hat{g}_t/\bar{g} + \hat{x}_t/\bar{x} - \hat{x}_{t-1}/\bar{x}\\
  \hat{k}_t = ((1-\delta)/\bar{g})[\hat{k}_{t-1}-(\bar{k}/\bar{g})\hat{g}_t] + \hat{x}_t\\
  \hat{\lambda}_t/\bar{\lambda} + \hat{\imath}_t/\bar{i} + \hat{s}_t/\bar{s}  = E_t\hat{\lambda}_{t+1}/\bar{\lambda}+E_t\hat{g}_{t+1}/\bar{g}+E_t\hat{\pi}_{t+1}/\bar{\pi} \\
  \hat{q}_t = \hat{\lambda}_t/\bar{\lambda} - E_t\hat{\lambda}_{t+1}/\bar{\lambda} + (\beta/\bar{g})[E_t\hat{r}^k_{t+1} + (1-\delta)E_t\hat{q}_{t+1}] - E_t\hat{g}_{t+1}/\bar{g}\\
  \hat{x}^g_t = \hat{q}_t + \beta E_t\hat{x}^g_{t+1}\\
  \varphi\hat{\pi}^{gap}_t = \theta\hat{mc}_t+\beta\varphi E_t\hat{\pi}^{gap}_{t+1}\\
  \hat{g}_t = \sigma_g\varepsilon_{g,t}\\
  \hat{s}_t = \rho_s\hat{s}_{t-1} + \sigma_s\varepsilon_{s,t}\\
  \hat{mp}_t = \sigma_i\varepsilon_{i,t}
\end{gather}
Variables:$\{\hat{c},\hat{n},\hat{x},\hat{k},\hat{y^{gdp}},\hat{y},\hat{x^g},\hat{y^g},\hat{w},\hat{r^k},\hat{\pi},\hat{i},\hat{i^n},\hat{q},\hat{\lambda},\hat{g},\hat{s},\hat{mp}\}$\\
\begin{comment}
\pagebreak









\subsection{ART Model pre-capital}
\noindent Preferences:
\begin{gather*}
  E_0\textstyle\sum_{t=0}^\infty\beta^t [\log(c_t-hc^a_{t-1})-\chi n_t^{1+\eta}/(1+\eta)]
\end{gather*}

\noindent Budget Constraint:
\begin{gather*}
  c_t+b_t/(i_ts_t)=w_tn_t+b_{t-1}/\pi_t+d_t
\end{gather*}

\setcounter{equation}{0}
\noindent Equilibrium system (11 equations):
\small\begin{gather}
c_t = [1-\varphi(\pi_t^{gap}-1)^2/2]y_t\\
i_t^n=(i^n_{t-1})^{\rho_i}(\bar{\imath}(\pi^{gap}_t)^{\phi_\pi}(c_t/(\bar{g}c_{t-1}))^{\phi_c})^{1-\rho_i}\exp(\sigma_\nu\nu_t)\\
i_t=\max\{1,i_t^n\}\\
\lambda_t = c_t - hc^a_{t-1} \\
w_t = \chi n_t^\eta \lambda_t\\
1 =  \beta E_t[(\lambda_t/\lambda_{t+1})(s_ti_t/(\bar{\pi}\pi_{t+1}^{gap}))]\\
\varphi(\pi_t^{gap}-1)\pi_t^{gap} = 1-\theta + \theta w_t/z_t + \beta\varphi E_t[(\lambda_t/\lambda_{t+1})(\pi_{t+1}^{gap}-1)\pi_{t+1}^{gap}(y_{t+1}/y_t)]\\
  y_t=z_t n_t\\  
  g_t= (1-\rho_g)\bar{g}+\rho_gg_{t-1} + \sigma_\varepsilon\varepsilon_t \\
  s_t=(1-\rho_s)\bar{s}+\rho_ss_{t-1} + \sigma_\upsilon\upsilon_t\\
  z_t=g_tz_{t-1}
\end{gather}\normalsize
Variables: $\{c,i^n,i,\lambda,w,\pi^{gap},y,n,g,s,z\}$\\

\setcounter{equation}{0}
\noindent De-trended Equilibrium System (10 equations):
\small\begin{gather}
\tilde{c}_t = [1-\varphi(\pi_t^{gap} - 1)^2/2]\tilde{y}_t\\
i_t^n=(i^n_{t-1})^{\rho_i}(\bar{\imath}(\pi_t^{gap})^{\phi_\pi}(g_t\tilde{c}_t/(\bar{g}\tilde{c}_{t-1})^{\phi_c})^{1-\rho_i}\exp(\sigma_\nu\nu_t)\\
i_t=\max\{1,i_t^n\}\\
\tilde{\lambda}_t = \tilde{c}_t\ - h\tilde{c}_{t-1}/g_t\\
\tilde{w}_t = \chi n_t^\eta \tilde{\lambda}_t  \\
  1 = \beta E_t[(\tilde{\lambda}_t/\tilde{\lambda}_{t+1})(s_ti_t/(\bar{\pi}\pi_{t+1}^{gap}g_{t+1}))]\\
  \varphi(\pi_t^{gap}-1){\pi}_t^{gap} = 1-\theta + \theta\tilde{w}_t + \beta\varphi E_t[(\tilde{\lambda}_t/\tilde{\lambda}_{t+1}) (\pi_{t+1}^{gap}-1)\pi_{t+1}^{gap}(\tilde{y}_{t+1}/\tilde{y}_t)]\\
  \tilde{y}_t= n_t\\  
  g_t= (1-\rho_g)\bar{g}+\rho_gg_{t-1} + \sigma_\varepsilon\varepsilon_t \\
  s_t=(1-\rho_s)s_t+\rho_ss_{t-1} + \sigma_\upsilon\upsilon_t
\end{gather}
Variables: $\{\tilde{c},i^n,i,\tilde{\lambda},\tilde{w},\pi^{gap},\tilde{y},n,k,g,s\}$\\ 

\setcounter{equation}{0}
\noindent Log-linear Equilibrium System:
\begin{gather}
  \hat{c}_t = \hat{y}_t\\
  \hat{\imath}_t^n = \rho_i\hat{\imath}^n_{t-1} + (1-\rho_i)\phi_\pi\hat{\pi}_t+ (1-\rho_i)\phi_c(\hat{g}+\hat{c}-\hat{c}_{t-1})+\sigma_\nu\nu_t \\
  \hat{\imath}_t = \hat{\imath}_t^n\\
  (1-h/g)\hat{\lambda}_t = \hat{c}_t + (h/g)(\hat{g} - \hat{c}_{t-1})\\
  \hat{w}_t =  \eta\hat{n}_t + \hat{\lambda}_t\\
  \hat{\lambda}_t + \hat{\imath}_t + s_t  = E_t\hat{\lambda}_{t+1}+E_t\hat{\pi}_{t+1} \\
  \varphi\hat{\pi}_t = (\theta-1)\hat{w}_t+\beta\varphi E_t\hat{\pi}_{t+1}\\
  \hat{y}_t = \hat{n}_t 
\end{gather}
\pagebreak
\setcounter{equation}{0}
\subsection{Gust Et Al Model}
\noindent Equilibrium system (13 equations):
\small\begin{gather}
\varphi(\pi_t^{gap}-1)\pi_t^{gap} = 1-\theta + \theta w_t/z_t + \beta\varphi E_t[(\lambda_t/\lambda_{t+1})(\pi_{t+1}^{gap}-1)\pi_{t+1}^{gap}(y_{t+1}/y_t)]\\
i_t^n=(i^n_{t-1})^{\rho_i}(\bar{\imath}(\pi^{gap}_t)^{\phi_\pi})^{1-\rho_i}\exp(\sigma_\nu\nu_t)\\
i_t=\max\{1,i_t^n\}\\
1/\lambda_t =  \beta E_t[(1/\lambda_{t+1})(s_ti_t/(\bar{\pi}\pi_{t+1}^{gap}g_{t+1}))]\\
\lambda_t = c_t \\
c_t = [1-\varphi(\pi_t^{gap}-1)^2/2]y_t\\
w_t = \chi n_t^\eta \lambda_t\\
  y_t=z_t n_t\\  
  g_t= (1-\rho_g)\bar{g}+\rho_gg_{t-1} + \sigma_\varepsilon\varepsilon_t \\
  s_t=(1-\rho_s)\bar{s}+\rho_ss_{t-1} + \sigma_\upsilon\upsilon_t\\
  z_t=g_tz_{t-1}
\end{gather}\normalsize
Variables: $\{c,i^n,i,\lambda,w,\pi^{gap},V_{\lambda},y,V_{\pi},n,g,s,z\}$\\

\setcounter{equation}{0}
\noindent De-trended Equilibrium System (12 equations):
\small\begin{gather}
\varphi(\pi_t^{gap}-1){\pi}_t^{gap} = 1 - \theta + \theta\tilde{w}_t + \beta\varphi E_t[(\tilde{\lambda}_t/\tilde{\lambda}_{t+1})(\pi^{gap}_{t+1}-1)\pi^{gap}_{t+1}(\tilde{y}_{t+1}/\tilde{y}_t)]\\
i_t^n=(i^n_{t-1})^{\rho_i}(\bar{\imath}(\pi_t^{gap})^{\phi_\pi})^{1-\rho_i}\exp(\sigma_\nu\nu_t)\\
i_t=\max\{1,i_t^n\}\\
1/\tilde{\lambda}_t = \beta E_t[(1/\tilde{\lambda}_{t+1})(s_ti_t/(\bar{\pi}\pi^{gap}_{t+1}g_{t+1}))]\\%
\tilde{\lambda}_t = \tilde{c}_t\\
\tilde{c}_t = [1-\varphi(\pi_t^{gap} - 1)^2/2]\tilde{y}_t\\
\tilde{w}_t = \chi n_t^\eta \tilde{\lambda}_t  \\
  \tilde{y}_t= n_t\\  
  g_t= (1-\rho_g)\bar{g}+\rho_gg_{t-1} + \sigma_\varepsilon\varepsilon_t \\
  s_t=(1-\rho_s)s_t+\rho_ss_{t-1} + \sigma_\upsilon\upsilon_t
\end{gather}
Variables: $\{\tilde{c},i^n,i,\tilde{\lambda},\tilde{w},\pi^{gap},\tilde{V}_{\lambda},\tilde{y},\tilde{V}_{\pi},n,g,s\}$\\ 

\setcounter{equation}{0}
\noindent Log-linear Equilibrium System:
\begin{gather}
  \varphi\hat{\pi}_t = (\theta-1)\hat{w}_t+\beta\varphi E_t\hat{\pi}_{t+1}\\
  \hat{\imath}_t^n = \rho_i\hat{\imath}^n_{t-1} + (1-\rho_i)\phi_\pi\hat{\pi}_t+\sigma_\nu\nu_t \\
  \hat{\imath}_t = \hat{\imath}_t^n\\
    -\hat{\lambda}_t = \hat{\imath}_t + s_t - E_t\hat{\lambda}_{t+1} - E_t\hat{\pi}_{t+1} \\
  \hat{\lambda}_t = \hat{c}_t \\
  \hat{c}_t = \hat{y}_t\\
  \hat{w}_t =  \eta\hat{n}_t + \hat{\lambda}_t\\
  \hat{y}_t = \hat{n}_t
\end{gather}

\setcounter{equation}{0}
\noindent Gust et al Indicator Functions\\
\begin{gather}
  c_{t+1,1} = \beta E_t[\lambda_t(s_ti_t/(\bar{\pi}\pi_{t+1}^{gap}g_{t+1}))]\\
  c_{t+1,2} = \beta E_t[\lambda_t(s_t/(\bar{\pi}\pi_{t+1}^{gap}g_{t+1}))]\\  
\end{gather}
for $k=1,2$ where $k=1$ corresponds to the non-ZLB regime and $k=2$ corresponds to the ZLB regime.
\begin{gather}
c_{t} = c_{t,1}I + c_{t,2}(1-I) 
\end{gather}
for $j=1,2$ and where $I$ is defined by:
$$
\begin{cases}
  I = 1 & \text{ if } i > 1\\
  I = 0 & \text{ otherwise.}
\end{cases}
$$
%We use the functions, $V_l$, to determine the decisions rule though we do not approximate $V_l$ directly, because they inherit a kink associated with the ZLB constraint. Instead, we approximate functions, $V_{l,i}$, that are smoother and easier to approximate by specifying:
%\begin{gather}
%V_l = V_{l,1}I + V_{l,2}(1-I) 
%\end{gather}
%for $l \in \{\lambda, \pi\}$ and $j=1,2$ and where $I$ is defined by:
%\begin{gather}
%  I = 1 \text{ if } i > 1\\
%  \text{  } = 0 \text{ otherwise}
%\end{gather}
%In the above, $i = \max(1,i^n)$ where $i^n$ denotes the value of the notional rate derived from evaluating the functions $V_{l,1}$ and using (2). (For each variable, we use $j=1$ to denote the function associated with the regime with a positive nominal interest rate and $j=2$ to denote the function associated with the ZLB regime. \\
%Because the functions, $V_l$ depend directly on the nominal interest rate, we expect then to have a kink or non-differentiability. By contrast, the counterpart functions, $V_{l,j}$, that are indexed by the interest-rate regime do not depend on the current indicator function and thus are more likely to be smooth. \\
%Specifically, $V_{\pi,1}$ and $V_{\pi,2}$ are both described by (8) because the nominal interest rate does not appear in that equation. $V_{\lambda,1}$ is described by (6), but for the ZLB regime:
%\begin{gather}
%V_{\lambda,t,2} =  \beta E_t[(1/\lambda_{t+1})(s_t/(\bar{\pi}\pi_{t+1}^{gap}g_{t+1}))]\\  
%\end{gather}
%where $i_t$ is replaced by 1.
\end{comment}
\end{document}

\documentclass[12pt, final]{article}
\usepackage{color}
\usepackage{times}
\usepackage{amssymb,amsmath,amsthm}
\usepackage{dsfont}
\usepackage{graphicx}
\usepackage{enumerate}
\usepackage{enumitem}
\usepackage[paperwidth=8.5in,left=1.0in,right=1.0in,top=1.0in,bottom=1.0in,paperheight=11.0in]{geometry}
\usepackage[labelsep=colon,singlelinecheck=false,footnotesize]{caption}
\usepackage{fancyhdr}
\usepackage{float}
\usepackage{booktabs}
\usepackage[sort&compress]{natbib}
\usepackage{subfig}
\usepackage{titlesec}
\usepackage[breaklinks]{hyperref}
\hypersetup{pdfdisplaydoctitle=true,bookmarksnumbered=true,colorlinks=true,citecolor=black,linkcolor=darkblue,urlcolor=darkred,pdfstartview=FitH,pdfpagemode=UseNone}
\usepackage[hyphenbreaks]{breakurl}
\usepackage{accents}
\newcommand{\ubar}[1]{\underaccent{\bar}{#1}}


%Define Color Names
\definecolor{Gray}{rgb}{0.65,0.65,0.65}
\definecolor{darkblue}{rgb}{0,0,0.55}
\definecolor{darkred}{rgb}{0.5,0,0}

%Section Headings
\newcommand\Bheadfont{\fontsize{14pt}{\baselineskip}\selectfont}
\titleformat{\section}[hang]{\normalfont\sc\color{darkblue}\Bheadfont}{\thesection\hskip0.618em}{0em}{}
\titlespacing*{\section}{0pt}{15pt plus 2pt minus 2pt}{9pt plus 2pt minus 2pt}
\titleformat{\subsection}[runin]{\normalfont\sc\color{darkblue}} {\thesubsection\hskip0.618em}{0em}{}
\titlespacing*{\subsection}{0pt}{13pt plus 2pt minus 2pt}{13pt plus 2pt minus 2pt}
\titleformat{\subsubsection}[runin]{\normalfont\sc\color{darkblue}} {\thesubsubsection\hskip0.618em}{0em}{}
\titlespacing*{\subsubsection}{0pt}{13pt plus 2pt minus 2pt}{13pt plus 2pt minus 2pt}
\titleformat{\paragraph}[runin]{\bfseries}{\theparagraph\hskip0.618em}{0em}{}
\titlespacing*{\paragraph}{0pt}{13pt plus 2pt minus 2pt}{13pt plus 2pt minus 2pt}

\begin{document}

\section{Equilibrium System}

\noindent Preferences:
\begin{gather*}
  E_0\textstyle\sum_{t=0}^\infty\beta^t [\log(c_t)-\chi n_t^{1+\eta}/(1+\eta)]
\end{gather*}

\noindent Budget Constraint:
\begin{gather*}
  c_t+b_t/(i_ts_t)=w_tn_t+b_{t-1}/\pi_t+d_t
\end{gather*}

\setcounter{equation}{0}
\subsection{ART Model}
\noindent Equilibrium system (11 equations):
\small\begin{gather}
c_t = [1-\varphi(\pi_t^{gap}-1)^2/2]y_t\\
i_t^*=(i^*_{t-1})^{\rho_i}(\bar{\imath}(\pi^{gap}_t)^{\phi_\pi})^{1-\rho_i}\exp(\sigma_\nu\nu_t)\\
i_t=\max\{1,i_t^*\}\\
\lambda_t = c_t \\
w_t = \chi n_t^\eta \lambda_t\\
1 =  \beta E_t[(\lambda_t/\lambda_{t+1})(s_ti_t/(\bar{\pi}\pi_{t+1}^{gap}g_{t+1}))]\\
\varphi(\pi_t^{gap}-1)\pi_t^{gap} = 1-\theta + \theta w_t/z_t + \beta\varphi E_t[(\lambda_t/\lambda_{t+1})(\pi_{t+1}^{gap}-1)\pi_{t+1}^{gap}(y_{t+1}/y_t)]\\
  y_t=z_t n_t\\  
  g_t= (1-\rho_g)\bar{g}+\rho_gg_{t-1} + \sigma_\varepsilon\varepsilon_t \\
  s_t=(1-\rho_s)\bar{s}+\rho_ss_{t-1} + \sigma_\upsilon\upsilon_t\\
  z_t=g_tz_{t-1}
\end{gather}\normalsize
Variables: $\{c,i^*,i,\lambda,w,\pi^{gap},y,n,g,s,z\}$\\

\setcounter{equation}{0}
\noindent De-trended Equilibrium System (10 equations):
\small\begin{gather}
\tilde{c}_t = [1-\varphi(\pi_t^{gap} - 1)^2/2]\tilde{y}_t\\
i_t^*=(i^*_{t-1})^{\rho_i}(\bar{\imath}(\pi_t^{gap})^{\phi_\pi})^{1-\rho_i}\exp(\sigma_\nu\nu_t)\\
i_t=\max\{1,i_t^*\}\\
\tilde{\lambda}_t = \tilde{c}_t\\
\tilde{w}_t = \chi n_t^\eta \tilde{\lambda}_t  \\
  1 = \beta E_t[(\tilde{\lambda}_t/\tilde{\lambda}_{t+1})(s_ti_t/(\bar{\pi}\pi_{t+1}^{gap}g_{t+1}))]\\
  \varphi(\pi_t^{gap}-1){\pi}_t^{gap} = 1-\theta + \theta\tilde{w}_t + \beta\varphi E_t[(\tilde{\lambda}_t/\tilde{\lambda}_{t+1}) (\pi_{t+1}^{gap}-1)\pi_{t+1}^{gap}(\tilde{y}_{t+1}/\tilde{y}_t)]\\
  \tilde{y}_t= n_t\\  
  g_t= (1-\rho_g)\bar{g}+\rho_gg_{t-1} + \sigma_\varepsilon\varepsilon_t \\
  s_t=(1-\rho_s)s_t+\rho_ss_{t-1} + \sigma_\upsilon\upsilon_t
\end{gather}
Variables: $\{\tilde{c},i^*,i,\tilde{\lambda},\tilde{w},\pi^{gap},\tilde{y},n,g,s\}$\\ 

\setcounter{equation}{0}
\noindent Log-linear Equilibrium System:
\begin{gather}
  \hat{c}_t = \hat{y}_t\\
  \hat{\imath}_t^n = \rho_i\hat{\imath}^n_{t-1} + (1-\rho_i)\phi_\pi\hat{\pi}_t+\sigma_\nu\nu_t \\
  \hat{\imath}_t = \hat{\imath}_t^n\\
  \hat{\lambda}_t = \hat{c}_t \\
  \hat{w}_t =  \eta\hat{n}_t + \hat{\lambda}_t\\
  \hat{\lambda}_t + \hat{\imath}_t + s_t  = E_t\hat{\lambda}_{t+1}+E_t\hat{\pi}_{t+1} \\
  \varphi\hat{\pi}_t = (\theta-1)\hat{w}_t+\beta\varphi E_t\hat{\pi}_{t+1}\\
  \hat{y}_t = \hat{n}_t 
\end{gather}
\pagebreak
\setcounter{equation}{0}
\subsection{Gust Et Al Model}
\noindent Equilibrium system (13 equations):
\small\begin{gather}
\varphi(\pi_t^{gap}-1)\pi_t^{gap} = 1-\theta + \theta w_t/z_t + \beta\varphi E_t[(\lambda_t/\lambda_{t+1})(\pi_{t+1}^{gap}-1)\pi_{t+1}^{gap}(y_{t+1}/y_t)]\\
i_t^*=(i^*_{t-1})^{\rho_i}(\bar{\imath}(\pi^{gap}_t)^{\phi_\pi})^{1-\rho_i}\exp(\sigma_\nu\nu_t)\\
i_t=\max\{1,i_t^*\}\\
1/\lambda_t =  \beta E_t[(1/\lambda_{t+1})(s_ti_t/(\bar{\pi}\pi_{t+1}^{gap}g_{t+1}))]\\
\lambda_t = c_t \\
c_t = [1-\varphi(\pi_t^{gap}-1)^2/2]y_t\\
w_t = \chi n_t^\eta \lambda_t\\
  y_t=z_t n_t\\  
  g_t= (1-\rho_g)\bar{g}+\rho_gg_{t-1} + \sigma_\varepsilon\varepsilon_t \\
  s_t=(1-\rho_s)\bar{s}+\rho_ss_{t-1} + \sigma_\upsilon\upsilon_t\\
  z_t=g_tz_{t-1}
\end{gather}\normalsize
Variables: $\{c,i^*,i,\lambda,w,\pi^{gap},V_{\lambda},y,V_{\pi},n,g,s,z\}$\\

\setcounter{equation}{0}
\noindent De-trended Equilibrium System (12 equations):
\small\begin{gather}
\varphi(\pi_t^{gap}-1){\pi}_t^{gap} = 1 - \theta + \theta\tilde{w}_t + \beta\varphi E_t[(\tilde{\lambda}_t/\tilde{\lambda}_{t+1})(\pi^{gap}_{t+1}-1)\pi^{gap}_{t+1}(\tilde{y}_{t+1}/\tilde{y}_t)]\\
i_t^*=(i^*_{t-1})^{\rho_i}(\bar{\imath}(\pi_t^{gap})^{\phi_\pi})^{1-\rho_i}\exp(\sigma_\nu\nu_t)\\
i_t=\max\{1,i_t^*\}\\
1/\tilde{\lambda}_t = \beta E_t[(1/\tilde{\lambda}_{t+1})(s_ti_t/(\bar{\pi}\pi^{gap}_{t+1}g_{t+1}))]\\%
\tilde{\lambda}_t = \tilde{c}_t\\
\tilde{c}_t = [1-\varphi(\pi_t^{gap} - 1)^2/2]\tilde{y}_t\\
\tilde{w}_t = \chi n_t^\eta \tilde{\lambda}_t  \\
  \tilde{y}_t= n_t\\  
  g_t= (1-\rho_g)\bar{g}+\rho_gg_{t-1} + \sigma_\varepsilon\varepsilon_t \\
  s_t=(1-\rho_s)s_t+\rho_ss_{t-1} + \sigma_\upsilon\upsilon_t
\end{gather}
Variables: $\{\tilde{c},i^*,i,\tilde{\lambda},\tilde{w},\pi^{gap},\tilde{V}_{\lambda},\tilde{y},\tilde{V}_{\pi},n,g,s\}$\\ 

\setcounter{equation}{0}
\noindent Log-linear Equilibrium System:
\begin{gather}
  \varphi\hat{\pi}_t = (\theta-1)\hat{w}_t+\beta\varphi E_t\hat{\pi}_{t+1}\\
  \hat{\imath}_t^n = \rho_i\hat{\imath}^n_{t-1} + (1-\rho_i)\phi_\pi\hat{\pi}_t+\sigma_\nu\nu_t \\
  \hat{\imath}_t = \hat{\imath}_t^n\\
    -\hat{\lambda}_t = \hat{\imath}_t + s_t - E_t\hat{\lambda}_{t+1} - E_t\hat{\pi}_{t+1} \\
  \hat{\lambda}_t = \hat{c}_t \\
  \hat{c}_t = \hat{y}_t\\
  \hat{w}_t =  \eta\hat{n}_t + \hat{\lambda}_t\\
  \hat{y}_t = \hat{n}_t
\end{gather}

\setcounter{equation}{0}
\noindent Gust et al Indicator Functions\\
\begin{gather}
  c_{t+1,1} = \beta E_t[\lambda_t(s_ti_t/(\bar{\pi}\pi_{t+1}^{gap}g_{t+1}))]\\
  c_{t+1,2} = \beta E_t[\lambda_t(s_t/(\bar{\pi}\pi_{t+1}^{gap}g_{t+1}))]\\  
\end{gather}
for $k=1,2$ where $k=1$ corresponds to the non-ZLB regime and $k=2$ corresponds to the ZLB regime.
\begin{gather}
c_{t} = c_{t,1}I + c_{t,2}(1-I) 
\end{gather}
for $j=1,2$ and where $I$ is defined by:
$$
\begin{cases}
  I = 1 & \text{ if } i > 1\\
  I = 0 & \text{ otherwise.}
\end{cases}
$$
%We use the functions, $V_l$, to determine the decisions rule though we do not approximate $V_l$ directly, because they inherit a kink associated with the ZLB constraint. Instead, we approximate functions, $V_{l,i}$, that are smoother and easier to approximate by specifying:
%\begin{gather}
%V_l = V_{l,1}I + V_{l,2}(1-I) 
%\end{gather}
%for $l \in \{\lambda, \pi\}$ and $j=1,2$ and where $I$ is defined by:
%\begin{gather}
%  I = 1 \text{ if } i > 1\\
%  \text{  } = 0 \text{ otherwise}
%\end{gather}
%In the above, $i = \max(1,i^*)$ where $i^*$ denotes the value of the notional rate derived from evaluating the functions $V_{l,1}$ and using (2). (For each variable, we use $j=1$ to denote the function associated with the regime with a positive nominal interest rate and $j=2$ to denote the function associated with the ZLB regime. \\
%Because the functions, $V_l$ depend directly on the nominal interest rate, we expect then to have a kink or non-differentiability. By contrast, the counterpart functions, $V_{l,j}$, that are indexed by the interest-rate regime do not depend on the current indicator function and thus are more likely to be smooth. \\
%Specifically, $V_{\pi,1}$ and $V_{\pi,2}$ are both described by (8) because the nominal interest rate does not appear in that equation. $V_{\lambda,1}$ is described by (6), but for the ZLB regime:
%\begin{gather}
%V_{\lambda,t,2} =  \beta E_t[(1/\lambda_{t+1})(s_t/(\bar{\pi}\pi_{t+1}^{gap}g_{t+1}))]\\  
%\end{gather}
%where $i_t$ is replaced by 1.
\end{comment}
\end{document}

